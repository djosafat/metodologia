\documentclass[12pt]{article}
%%%%%%%%%%%%%%%%%%%%%%%
\usepackage[left=2.5cm,right=2.5cm,top=2cm,bottom=4cm]{geometry}
%%%% spanish %%%%%%%%%%
\usepackage[spanish,es-tabla]{babel}%hyphenation
\usepackage[utf8]{inputenc} % accents
\usepackage[T1]{fontenc} % encoding
\usepackage{lmodern} % font
%%%%%%%%%%%%%%%%%%%%%%%%
\usepackage{schemata} %Paquete para crear cuadros conceptuales
%%%%%%%%%%%%%%%%%%%%%%%%%%%%%%%%%%%%%%%%
\title{Tipos de fuentes}
\author{David Santana}
\date{October 2020}

\begin{document}

\maketitle

\section*{¿De dónde parte una investigación?}
Las fuentes bibliográficas, también llamadas referencias, son los documentos y trabajos consultados que dan fundamento a lo que escribimos o desarrollamos en una investigación. Las fuentes consultadas, por ejemplo libros, artículos científicos, páginas web, por mencionar algunas, deben cumplir con ciertos criterios de formalidad al haber sido sometidos a revisiones por pares. Lo anterior significa que uno o varios especialistas del tema debieron haber hecho un \emph{arbitraje} previo a la publicación del trabajo en cuestión. En el siguiente diagrama se puede ver una clasificación de acuerdo a los objetivos de cada fuente. Tal diagrama es enunciativo, mas no limitativo.
\vspace{1cm}

\schema{
    \schemabox{Tipos de fuentes}}{
        \schemabox{Libros}
        \schemabox{Artículos científicos}
        \schemabox{Páginas web}
}

\vspace{1cm}

\begin{center}
\begin{minipage}[c]{1\textwidth}
\schema{
    \schemabox{Tipos de fuentes}}{
        \schemabox{Libros}
        \schemabox{Artículos científicos}
        \schemabox{Páginas web}
}
\end{minipage}
\end{center}

\vspace{1cm}

\begin{center}
\begin{minipage}[c]{1\textwidth}
\schema{\schemabox{Tipos de fuentes}}{
        	\schemabox{
        		\schema{\schemabox{$\bullet$ Libros}}{
        				\schemabox{$\bullet$ Para apoyar un curso}
        				\schemabox{$\bullet$ De difusión}
        				\schemabox{$\bullet$ Científico}
        		}
    		}
    		\schemabox{Artículos científicos}
    		\schemabox{Páginas web}
}
\end{minipage}
\end{center}

\vspace{1cm}

\begin{center}
\begin{minipage}[c]{1\textwidth}
\schema{\schemabox{Tipos de fuentes}}{
        	\schemabox{
        		\schema{\schemabox{$\bullet$ Libros}}{
	        				\schemabox{$\bullet$ Para apoyar un curso}
    		    				\schemabox{$\bullet$ De difusión}
        					\schemabox{$\bullet$ Científico}
        		}
    		}
    		\schemabox{
    			\schema{\schemabox{$\bullet$ Artículos científicos}}{
    						\schemabox{$\bullet$ De difusión}
    						\schemabox{
    							\schema{\schemabox{$\bullet$ Con nuevos resultados}}{\schemabox{--Teóricos}\schemabox{--Experimentales}}}
    		}}
    		\schemabox{$\bullet$ Páginas web}
}
\end{minipage}
\end{center}

La principal diferencia entre un libro o artículo de difusión y los libros y artículos de investigación original, es que en el segundo caso se expande el conocimiento con resultados nuevos y se generan conclusiones a partir de experimentación. Tanto en libros como en artículos científicos se pueden encontrar trabajos teóricos y experimentales. 

\end{document}
