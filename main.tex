\documentclass{article} 
%\documentclass{book}
%%%%%%%%%%%%%%%%%%%%%%%
\usepackage[left=2.5cm,right=2.5cm,top=2cm,bottom=4cm]{geometry}
%%%% spanish %%%%%%%%%%
\usepackage[spanish]{babel}%hyphenation
%\usepackage[spanish,es-tabla]{babel}%hyphenation
\usepackage[utf8]{inputenc} % accents
\usepackage[T1]{fontenc} % encoding
\usepackage{lmodern} % font
\usepackage{array} %alineación vertical p (top) m (middle) b (bottom)
\usepackage{soul} %separación entre letras
\usepackage{booktabs}
%%%%%%%%%%%%%%%%%%%%%%%%%%%%%%%%%%%%%%%%%%%%%%%%
\title{Tablas (cuadros) en \LaTeX}
\author{David Santana}
\date{DACB UJAT}
%%%%%%%%%%%%%%%%%%%%%%%%%%%%%%%%%%%%%%%%%%%%%%%%

\begin{document}
\maketitle

\section{Horario de un estudiante}

Un alumno de bachillerato está preocupado por cumplir con sus labores académica. Sus clases, que este semestre serán en aula virtual, comienzan a las 8 a.m. y terminan a las 13 horas. Además, incluyó otras actividades integrales en su formación. Por ejemplo, tuvo en cuenta programar a qué hora hacer ejercicio. El horario que el alumno configuró se puede observar en el Cuadro \ref{tabla-horario-1}.

\begin{table}[h] %h de 'here'
\centering
\begin{tabular}{|c||ccccc|}
\hline
Horario & Lunes  & Martes & Miércoles & Jueves & Viernes \\ 
\hline \hline 
5 a 6   & Correr & Correr & Correr    & Correr & Correr\\
\hline 
6 a 7   & Bañarse & Bañarse & Bañarse    & Bañarse & Bañarse\\
\hline 
7 a 8   & Desayunar & Desayunar & Desayunar & Desayunar & Desayunar\\
\hline 
8 a 9   & Matemáticas & Matemáticas & Matemáticas & Matemáticas y Otros & Matemáticas\\
\hline 
9 a 10   & Historia & Historia & Historia & Historia & Historia\\
\hline 
10 a 11   & Biología & Biología & Biología & Biología & Biología\\
\hline 
11 a 12   & Español & Español & Español & Español & Español\\
\hline 
12 a 13   & Laboratorio & Física & Química    & Física & Química\\
\hline 
\end{tabular} 
\caption{Esta tabla muestra el horario relacionado a las actividades de un estudiante de bachillerato.}\label{tabla-horario-1}
%\vspace{0.5cm}
\end{table}

Cuando terminó, observó que podía hacer varias mejoras y diseñó un segundo horario, véase Cuadro \ref{tabla-horario-2}.
\begin{table}[h] %h de 'here'
\centering
\begin{tabular}{m{2cm} m{2cm} m{2cm} m{2cm} m{2cm} m{2cm}}
\hline
\textbf{Horario} & \textbf{Lunes}  & \textbf{Martes} & \textbf{Miércoles} & \textbf{Jueves} & \textbf{Viernes} \\ 
\hline  
5 a 6   & Correr todos los días & Correr & Correr    & Correr & Correr\\
\hline 
6 a 7   & Bañarse & Bañarse & Bañarse    & Bañarse & Bañarse\\
\hline 
7 a 8   & Desayunar & Desayunar & Desayunar & Desayunar & Desayunar\\
\hline 
8 a 9   & Matemáticas & Matemáticas & Matemáticas & Matemáticas & Matemáticas\\
\hline 
9 a 10   & Historia & Historia & Historia & Historia & Historia\\
\hline 
10 a 11   & Biología & Biología & Biología & Biología & Biología\\
\hline 
11 a 12   & Español & Español & Español & Español & Español\\
\hline 
12 a 13   & Laboratorio & Física & Química    & Física & Química\\
\hline 
\end{tabular} 
\caption{Esta tabla muestra el horario relacionado a las actividades de un estudiante de bachillerato.}\label{tabla-horario-2}
%\vspace{0.5cm}
\end{table}
Al último pensó que se repetían demasiado los nombres de las actividades todos los días y llegó a la versión final. Véase Cuadro \ref{tabla-horario-3}.
\begin{table}[h] %h de 'here'
\centering
\begin{tabular}{m{2cm}  c >{\centering}m{2cm} c >{\centering}m{2cm} c >{\centering}m{2cm} c >{\centering}m{2cm} c >{\centering}m{2cm} } % solo con \usepackage{array}
\hline
\textbf{Horario del semestre} & \textbf{Lunes}  & \textbf{Martes} & \textbf{Miércoles} & \textbf{Jueves} & \textbf{Viernes} \\ 
\hline  
5 a 6   & \multicolumn{5}{c}{C\,\quad o\,\quad r\,\quad r\,\quad e\,\quad  r} \\
\hline 
6 a 7   & \multicolumn{5}{c}{B\,\qquad a\,\qquad ñ\,\qquad a\,\qquad r\,\qquad s\,\qquad e}\\
\hline 
7 a 8   & \multicolumn{5}{c}{Des\,\,\,\,\,\,\,\,\,\,\,\,ayunar}\\
\hline 
8 a 9   & \multicolumn{5}{c}{Ma\quad \quad \quad \quad \quad temáticas}\\
\hline 
9 a 10   & \multicolumn{5}{c}{ \so{Historia} }\\
\hline 
10 a 11   & \multicolumn{5}{c}{Bio\qquad \qquad \qquad \qquad \qquad \qquad logía}\\
\hline 
11 a 12   & \multicolumn{5}{c}{ Es\hspace{5cm}pañol }\\
\hline 
12 a 13   & Laboratorio & Física & Química    & Física & Química\\
\hline 
\end{tabular} 
\caption{Esta tabla muestra el horario relacionado a las actividades de un estudiante de bachillerato.}\label{tabla-horario-3}
%\vspace{0.5cm}
\end{table}

\begin{table}[h]
    \centering
    \begin{tabular}{cc}
        \toprule
         \multicolumn{2}{c}{Última tabla}\\
        \midrule
        ¿Se pueden hacer tablas más elegantes? & Sí \\
        \midrule
         ¿Con qué paquete? & booktabs \\
         \midrule
         ¿Se pueden hacer filas múltiples? & Sí \\
         \midrule
         ¿Con qué paquete? & multirow \\
        \bottomrule
    \end{tabular}
    %\caption{Caption}
    %\label{loquesea}
\end{table}

\end{document}
